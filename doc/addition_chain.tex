\documentclass{article}
\usepackage[utf8]{inputenc}
\usepackage{amsmath}
\usepackage{amsfonts}
\usepackage{amssymb}
\usepackage{bm}
\usepackage{url}
\usepackage{algorithmic}
\usepackage[margin=2.5cm]{geometry}

%opening
\title{Multiplicative Inverse}
\author{Armando Faz\\\footnotesize{\url{armfazh@ic.unicamp.br}}}
\date{}
\begin{document}
\maketitle

\section{Itoh-Tsujii}
 In order to compute the multiplicative inverse 
of an element $a\in \mathbb{F}_p^*$, the following identity is used: $a^{-1}\equiv 
a^{p-2}\; (\textrm{mod }p)$; part of this exponentiation can be calculated using an addition chain as shown by Itoh-Tsujii.

Let $x,y\in \mathbb{Z}^+$ and $x\leq y$, define the term $\bm{\alpha}_{x} = a^{2^x-1}$ and the relation $\bm{\alpha}_x\rightarrow \bm{\alpha}_y$ as $\bm{\alpha}_{y}=(\bm{\alpha}_x)^{2^{y-x}}\bm{\alpha}_{y-x}$.

The cost for $\bm{\alpha}_x\rightarrow \bm{\alpha}_y$ is $y-x$ squares and 1 multiplication.


\subsection{$p=2^{384}-2^{128}-2^{96}+2^{32}-1.$}
This is the addition chain for NIST P-384 curve. We start with $
\bm{\alpha}_{1} \rightarrow \bm{\alpha}_{2}  \rightarrow \bm{\alpha}_{3}  \rightarrow 
\bm{\alpha}_{6} \rightarrow \bm{\alpha}_{12} \rightarrow \bm{\alpha}_{15} \rightarrow 
\bm{\alpha}_{30}$. After that, we pre-compute $T_1 = a^{2^{32}-1}$, $T_2 = a^{2^{32}-3}$, $T_3 = a^{2^{31}-1}$.
using the following:
\begin{eqnarray*}
 T_1 = a^{2^{32}-1} &=& (\bm{\alpha}_{30})^{2^2}a \\
 T_2 = a^{2^{32}-3} &=& (\bm{\alpha}_{30})^{2^2}\bm{\alpha}_2 \\
 T_3 = a^{2^{31}-1} &=& (\bm{\alpha}_{30})^{2}a 
\end{eqnarray*}
Then, define  $L_{0},\dots,L_{353}= 0$ and 
\begin{eqnarray*}
 a^{T_2} &=& L_{0  } \\
 a^{T_1} &=& L_{96 } = 
 L_{129} = 
 L_{161} = 
 L_{193} = 
 L_{225} = 
 L_{257} = 
 L_{289} = 
 L_{321} \\
 a^{T_3} &=& L_{353}
\end{eqnarray*}
To obtain $c =a^{-1}$ compute:
\begin{algorithmic}
 \STATE $c = 1$
 \FOR{$i=353$ \TO 0}
	\STATE $c= c^2$
	\IF{$L_i \neq 0$}
		\STATE  $c = c\times L_i$
	\ENDIF
 \ENDFOR
 \RETURN $c$
\end{algorithmic}
Cost: 19 multiplications and 384 squares.

\subsection{$p=2^{255}-19.$}

It was given an addition chain for $\mathbb{F}_{2^{255}-19}$, starting with $a^9=(a^2)^{2^2}\cdot a$ and $\bm{\alpha}_5=a^{31}=(a^2\cdot a^9)^2\cdot (a^9)$  compute   
$
\bm{\alpha}_{5} \rightarrow \bm{\alpha}_{10} \rightarrow \bm{\alpha}_{20} \rightarrow 
\bm{\alpha}_{40} \rightarrow \bm{\alpha}_{50} \rightarrow \bm{\alpha}_{100} \rightarrow 
\bm{\alpha}_{200} \rightarrow \bm{\alpha}_{250}
$, the multiplicative inverse is obtained as 
\begin{eqnarray*}
a^{-1} &=& a^{2^{255}-21} \\
       &=& (\bm{\alpha}_{250})^{2^5}\cdot a^{11}
\end{eqnarray*}
Cost: 11 multiplications and 254 squares.


\subsection{$p=2^{448}-2^{224}-1.$} For $\mathbb{F}_{2^{448}-2^{224}-1}$, starting with  
$
\bm{\alpha}_{1} \rightarrow   \bm{\alpha}_{2}  \rightarrow
\bm{\alpha}_{3}   \rightarrow \bm{\alpha}_{6}  \rightarrow \bm{\alpha}_{12} \rightarrow 
\bm{\alpha}_{24}  \rightarrow \bm{\alpha}_{27} \rightarrow \bm{\alpha}_{54} \rightarrow 
\bm{\alpha}_{108} \rightarrow \bm{\alpha}_{111}\rightarrow \bm{\alpha}_{222} \rightarrow 
\bm{\alpha}_{223}
$, the multiplicative inverse is obtained as $a^{-1} = b^{2^2}\cdot a$:
\begin{eqnarray*}
b &=& (\bm{\alpha}_{223})^{2^{223}}\cdot (\bm{\alpha}_{222})\\
  &=& (a^{2^{223}}-1)^{2^{223}} (a^{2^{222}}-1)\\
  &=& a^{2^{446}-2^{223}+2^{222}-1}\\
  &=& a^{2^{446}-2^{222}-1}
\end{eqnarray*}
\begin{eqnarray*}
a^{-1} &=& b^{2^2}\cdot a\\
	   &=& (a^{2^{446}-2^{222}-1})^{2^2}a\\
	   &=& a^{2^{448}-2^{222}-4+1}\\
       &=& a^{2^{448}-2^{224}-3}\\
\end{eqnarray*}

Cost: 13 multiplications and 447 squares.

\paragraph{Bonus:}  The value $b=2^{446}-2^{222}-1$ is used for computing square-roots ($b = \frac{p-3}{4}$); thus, there is only one addition chain used for both purposes.

\end{document}
